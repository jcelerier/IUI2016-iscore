\documentclass{sigchi}

% Use this command to override the default ACM copyright statement (e.g. for preprints). 
% Consult the conference website for the camera-ready copyright statement.


%% EXAMPLE BEGIN -- HOW TO OVERRIDE THE DEFAULT COPYRIGHT STRIP -- (July 22, 2013 - Paul Baumann)
% \toappear{Permission to make digital or hard copies of all or part of this work for personal or classroom use is 	granted without fee provided that copies are not made or distributed for profit or commercial advantage and that copies bear this notice and the full citation on the first page. Copyrights for components of this work owned by others than ACM must be honored. Abstracting with credit is permitted. To copy otherwise, or republish, to post on servers or to redistribute to lists, requires prior specific permission and/or a fee. Request permissions from permissions@acm.org. \\
% {\emph{CHI'14}}, April 26--May 1, 2014, Toronto, Canada. \\
% Copyright \copyright~2014 ACM ISBN/14/04...\$15.00. \\
% DOI string from ACM form confirmation}
%% EXAMPLE END -- HOW TO OVERRIDE THE DEFAULT COPYRIGHT STRIP -- (July 22, 2013 - Paul Baumann)


% Arabic page numbers for submission. 
% Remove this line to eliminate page numbers for the camera ready copy
% \pagenumbering{arabic}


% Load basic packages
\usepackage{balance}  % to better equalize the last page
\usepackage{graphics} % for EPS, load graphicx instead
\usepackage{times}    % comment if you want LaTeX's default font
\usepackage{url}      % llt: nicely formatted URLs

% llt: Define a global style for URLs, rather that the default one
\makeatletter
\def\url@leostyle{%
  \@ifundefined{selectfont}{\def\UrlFont{\sf}}{\def\UrlFont{\small\bf\ttfamily}}}
\makeatother
\urlstyle{leo}


% To make various LaTeX processors do the right thing with page size.
\def\pprw{8.5in}
\def\pprh{11in}
\special{papersize=\pprw,\pprh}
\setlength{\paperwidth}{\pprw}
\setlength{\paperheight}{\pprh}
\setlength{\pdfpagewidth}{\pprw}
\setlength{\pdfpageheight}{\pprh}

% Make sure hyperref comes last of your loaded packages, 
% to give it a fighting chance of not being over-written, 
% since its job is to redefine many LaTeX commands.
\usepackage[pdftex]{hyperref}
\hypersetup{
pdftitle={SIGCHI Conference Proceedings Format},
pdfauthor={LaTeX},
pdfkeywords={SIGCHI, proceedings, archival format},
bookmarksnumbered,
pdfstartview={FitH},
colorlinks,
citecolor=black,
filecolor=black,
linkcolor=black,
urlcolor=black,
breaklinks=true,
}

% create a shortcut to typeset table headings
\newcommand\tabhead[1]{\small\textbf{#1}}


% End of preamble. Here it comes the document.
\begin{document}

\title{SIGCHI Conference Proceedings Format}

\numberofauthors{3}
\author{
  \alignauthor 1st Author Name\\
    \affaddr{Affiliation}\\
    \affaddr{Address}\\
    \email{e-mail address}\\
    \affaddr{Optional phone number}
  \alignauthor 2nd Author Name\\
    \affaddr{Affiliation}\\
    \affaddr{Address}\\
    \email{e-mail address}\\
    \affaddr{Optional phone number}    
  \alignauthor 3rd Author Name\\
    \affaddr{Affiliation}\\
    \affaddr{Address}\\
    \email{e-mail address}\\
    \affaddr{Optional phone number}
}

\maketitle

\begin{abstract}
  The development of interactive shows and interactive user interfaces for arts \& exhibitions
has traditionally been done with tools that pertain to two broad metaphors. 
Cue-based environments work by making groups of parameters and sending them to remote devices, 
while more interactive applications are generally written in domain-specific 
programming environments, like Max/MSP, Processing or OpenFrameworks.
  In this paper, we argue about the specific issues that arise in such environments, and we present 
i-score : an extensive and collaborative software suite that bridges
the gap between time-based, logic-based and flow-based interactive application authoring tools. 
This is done in a single cohesive graphical user interface, built upon a few simple and novel primitives.
  i-score allows the creation of software meant for operation in a large parameter space, 
and enables artists to express easily both temporal logic and structured programming, 
with facilities for automating and applying transformations to single and multi-dimensional parameters.
\end{abstract}

\keywords{
	Guides; instructions; author's kit; conference publications;
	keywords should be separated by a semi-colon. \newline
	\textcolor{red}{Optional section to be included in your final version, 
  but strongly encouraged.}
}

\category{H.5.m.}{Information Interfaces and Presentation (e.g. HCI)}{Miscellaneous}

See: \url{http://www.acm.org/about/class/1998/}
for more information and the full list of ACM classifiers
and descriptors. \newline
\textcolor{red}{Optional section to be included in your final version, 
but strongly encouraged. On the submission page only the classifiers’ 
letter-number combination will need to be entered.}

\section{Introduction}
This paper presents a paradigm that aims to allow non-programmers 
to conceive interactive applications easily and execute them in production.

The existing software stack is either oriented too much towards the 
cue paradigm, which is useful as long as there is no complex logic involved, 
or towards the programming paradigm, where it is hard to write simple scenarios 
like "move a spotlight in horizontal oscillation for ten seconds; after the first 5 
seconds, if a dancer jumps on the stage, play a sound and increase reverberation steadily as long as the dancer is on stage".
interactive applications easily for non-programmers, and 

\subsection{Motivation}

\subsection{Use cases}

\subsection{Existing approaches}

\subsubsection{Content creation}
Flash, domain-specific software...

\subsubsection{Flow control models}
 % du tout temporel au tout logique en passant par systèmes réactifs ?
Max, Processing, OF, PureData, React.[...], Integra Live (qui est plutôt orienté son), Unity \& envs de jeu, etc. (revoir slides), Chronic (cf. téléchargements), OpenMusic, Antescofo (et Ascograph), logiciels de la conférence sur appli réactives (cf. slack).

\subsubsection{Document models and application description} % bof ici
CORBA, DBus, DOM HTML, DOM Qt, DOM Jamoma...


\section{A model for orchestration}
We will present our constructs by starting with the purely temporal ones, 
and by extending to the more logical aspects.
\subsection{Specification of temporal relationships}
- Duration
- Synchronization

- Mettre le tableau qui montre les problèmes avec les graphes qui s'interposent (en partant des petits schémas)

\subsection{Structured temporal programming}
- Conditions and trigger of synchronizations

Building from it : 
- Loops (because interpreter works like a playhead).
-> parler des différentes possibilités que ça ouvre : 
* Canons
* Fractales
* etc..


\section{From structure to content}
- Processus
- Hiérarchie (qu'on a déjà présenté).

- Paramètres : actuellement, scope global
Puis scope local dans la hiérarchie ?
-> rejoint la programmation structurée via les données

Perspective : transformations appliquées à une boîte

- Questions de synchronisation d'états (avant / pendant / après)

- Passage de messages pour contrôle interne ou externe

- Automations  

- Mappings 1D

- Mappings n-D

- Conclusion : analogue à un petit OS spécialisé pour applications multimédia.

\section{Shortcomings (et pistes)}
\subsection{Debugging}
- Getting execution traces
-> How to go at any point in the flow of a software ? The external state might not be correct.
(Conoundrum of "instant debugging" but requiring for instance a smoke machine to spit smoke for ten minutes)

- Visualisation / simulation du résultat ? 

\section{Evaluation}
- Time to develop artistic installations greatly reduced.

\section{Conclusion}
%% Logiciel ouvert et utilisable (API C++)
%% Perspectives : autres implémentations (FPGA, kdbus, modèle de calcul ?)

\end{document}
